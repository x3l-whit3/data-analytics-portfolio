% Options for packages loaded elsewhere
\PassOptionsToPackage{unicode}{hyperref}
\PassOptionsToPackage{hyphens}{url}
\documentclass[
]{article}
\usepackage{xcolor}
\usepackage[margin=1in]{geometry}
\usepackage{amsmath,amssymb}
\setcounter{secnumdepth}{-\maxdimen} % remove section numbering
\usepackage{iftex}
\ifPDFTeX
  \usepackage[T1]{fontenc}
  \usepackage[utf8]{inputenc}
  \usepackage{textcomp} % provide euro and other symbols
\else % if luatex or xetex
  \usepackage{unicode-math} % this also loads fontspec
  \defaultfontfeatures{Scale=MatchLowercase}
  \defaultfontfeatures[\rmfamily]{Ligatures=TeX,Scale=1}
\fi
\usepackage{lmodern}
\ifPDFTeX\else
  % xetex/luatex font selection
\fi
% Use upquote if available, for straight quotes in verbatim environments
\IfFileExists{upquote.sty}{\usepackage{upquote}}{}
\IfFileExists{microtype.sty}{% use microtype if available
  \usepackage[]{microtype}
  \UseMicrotypeSet[protrusion]{basicmath} % disable protrusion for tt fonts
}{}
\makeatletter
\@ifundefined{KOMAClassName}{% if non-KOMA class
  \IfFileExists{parskip.sty}{%
    \usepackage{parskip}
  }{% else
    \setlength{\parindent}{0pt}
    \setlength{\parskip}{6pt plus 2pt minus 1pt}}
}{% if KOMA class
  \KOMAoptions{parskip=half}}
\makeatother
\usepackage{color}
\usepackage{fancyvrb}
\newcommand{\VerbBar}{|}
\newcommand{\VERB}{\Verb[commandchars=\\\{\}]}
\DefineVerbatimEnvironment{Highlighting}{Verbatim}{commandchars=\\\{\}}
% Add ',fontsize=\small' for more characters per line
\usepackage{framed}
\definecolor{shadecolor}{RGB}{248,248,248}
\newenvironment{Shaded}{\begin{snugshade}}{\end{snugshade}}
\newcommand{\AlertTok}[1]{\textcolor[rgb]{0.94,0.16,0.16}{#1}}
\newcommand{\AnnotationTok}[1]{\textcolor[rgb]{0.56,0.35,0.01}{\textbf{\textit{#1}}}}
\newcommand{\AttributeTok}[1]{\textcolor[rgb]{0.13,0.29,0.53}{#1}}
\newcommand{\BaseNTok}[1]{\textcolor[rgb]{0.00,0.00,0.81}{#1}}
\newcommand{\BuiltInTok}[1]{#1}
\newcommand{\CharTok}[1]{\textcolor[rgb]{0.31,0.60,0.02}{#1}}
\newcommand{\CommentTok}[1]{\textcolor[rgb]{0.56,0.35,0.01}{\textit{#1}}}
\newcommand{\CommentVarTok}[1]{\textcolor[rgb]{0.56,0.35,0.01}{\textbf{\textit{#1}}}}
\newcommand{\ConstantTok}[1]{\textcolor[rgb]{0.56,0.35,0.01}{#1}}
\newcommand{\ControlFlowTok}[1]{\textcolor[rgb]{0.13,0.29,0.53}{\textbf{#1}}}
\newcommand{\DataTypeTok}[1]{\textcolor[rgb]{0.13,0.29,0.53}{#1}}
\newcommand{\DecValTok}[1]{\textcolor[rgb]{0.00,0.00,0.81}{#1}}
\newcommand{\DocumentationTok}[1]{\textcolor[rgb]{0.56,0.35,0.01}{\textbf{\textit{#1}}}}
\newcommand{\ErrorTok}[1]{\textcolor[rgb]{0.64,0.00,0.00}{\textbf{#1}}}
\newcommand{\ExtensionTok}[1]{#1}
\newcommand{\FloatTok}[1]{\textcolor[rgb]{0.00,0.00,0.81}{#1}}
\newcommand{\FunctionTok}[1]{\textcolor[rgb]{0.13,0.29,0.53}{\textbf{#1}}}
\newcommand{\ImportTok}[1]{#1}
\newcommand{\InformationTok}[1]{\textcolor[rgb]{0.56,0.35,0.01}{\textbf{\textit{#1}}}}
\newcommand{\KeywordTok}[1]{\textcolor[rgb]{0.13,0.29,0.53}{\textbf{#1}}}
\newcommand{\NormalTok}[1]{#1}
\newcommand{\OperatorTok}[1]{\textcolor[rgb]{0.81,0.36,0.00}{\textbf{#1}}}
\newcommand{\OtherTok}[1]{\textcolor[rgb]{0.56,0.35,0.01}{#1}}
\newcommand{\PreprocessorTok}[1]{\textcolor[rgb]{0.56,0.35,0.01}{\textit{#1}}}
\newcommand{\RegionMarkerTok}[1]{#1}
\newcommand{\SpecialCharTok}[1]{\textcolor[rgb]{0.81,0.36,0.00}{\textbf{#1}}}
\newcommand{\SpecialStringTok}[1]{\textcolor[rgb]{0.31,0.60,0.02}{#1}}
\newcommand{\StringTok}[1]{\textcolor[rgb]{0.31,0.60,0.02}{#1}}
\newcommand{\VariableTok}[1]{\textcolor[rgb]{0.00,0.00,0.00}{#1}}
\newcommand{\VerbatimStringTok}[1]{\textcolor[rgb]{0.31,0.60,0.02}{#1}}
\newcommand{\WarningTok}[1]{\textcolor[rgb]{0.56,0.35,0.01}{\textbf{\textit{#1}}}}
\usepackage{graphicx}
\makeatletter
\newsavebox\pandoc@box
\newcommand*\pandocbounded[1]{% scales image to fit in text height/width
  \sbox\pandoc@box{#1}%
  \Gscale@div\@tempa{\textheight}{\dimexpr\ht\pandoc@box+\dp\pandoc@box\relax}%
  \Gscale@div\@tempb{\linewidth}{\wd\pandoc@box}%
  \ifdim\@tempb\p@<\@tempa\p@\let\@tempa\@tempb\fi% select the smaller of both
  \ifdim\@tempa\p@<\p@\scalebox{\@tempa}{\usebox\pandoc@box}%
  \else\usebox{\pandoc@box}%
  \fi%
}
% Set default figure placement to htbp
\def\fps@figure{htbp}
\makeatother
\setlength{\emergencystretch}{3em} % prevent overfull lines
\providecommand{\tightlist}{%
  \setlength{\itemsep}{0pt}\setlength{\parskip}{0pt}}
\usepackage{bookmark}
\IfFileExists{xurl.sty}{\usepackage{xurl}}{} % add URL line breaks if available
\urlstyle{same}
\hypersetup{
  pdftitle={Case Study 01},
  hidelinks,
  pdfcreator={LaTeX via pandoc}}

\title{Case Study 01}
\author{}
\date{\vspace{-2.5em}2025-12-05}

\begin{document}
\maketitle

\subsection{Case study documentation}\label{case-study-documentation}

I followed the APPASA model for this case study. I tried to do it by
myself following the steps as learned in the Coursera's Data Analytics
Google Course.

\subsection{Scenario context}\label{scenario-context}

Cyclistic is a bike-sharing company. They ask me to find how they casual
users and members differ in the usage of their products.

\subsection{Ask}\label{ask}

\subsubsection{Main goals and expectations from the
stakeholders}\label{main-goals-and-expectations-from-the-stakeholders}

\begin{itemize}
\item
  Would it be good for them to convert the most users to members?
\item
  What are the main differences in the way the casual users and the
  members use their services?
\end{itemize}

\subsection{Prepare}\label{prepare}

\subsubsection{Quick desciption of the datasets and
tools}\label{quick-desciption-of-the-datasets-and-tools}

For this project, I use the tidyverse suite of packages along with
ggplot2. The dataset consists of 12 month of historical trip data
provided through a Coursera link to Cyclistic's public data repository.
The data appears credible and covers variable such as ID, timestamps,
locations, rideable type, and user type (member or casual).

\subsection{Process}\label{process}

\subsubsection{Data cleaning and preparation
steps}\label{data-cleaning-and-preparation-steps}

In this section, I perform the necessary transformations to prepare the
data for analysis.

This includes:

\begin{itemize}
\tightlist
\item
  importing and combining the 12 monthly files,
\end{itemize}

\begin{Shaded}
\begin{Highlighting}[]
\FunctionTok{library}\NormalTok{(tidyverse)}
\FunctionTok{library}\NormalTok{(ggplot2)}

\NormalTok{dec }\OtherTok{\textless{}{-}} \FunctionTok{read\_csv}\NormalTok{(}\StringTok{\textquotesingle{}202511{-}divvy{-}tripdata.csv\textquotesingle{}}\NormalTok{)}
\NormalTok{nov }\OtherTok{\textless{}{-}} \FunctionTok{read\_csv}\NormalTok{(}\StringTok{\textquotesingle{}202510{-}divvy{-}tripdata.csv\textquotesingle{}}\NormalTok{)}
\NormalTok{oct }\OtherTok{\textless{}{-}} \FunctionTok{read\_csv}\NormalTok{(}\StringTok{\textquotesingle{}202509{-}divvy{-}tripdata.csv\textquotesingle{}}\NormalTok{)}
\NormalTok{sep }\OtherTok{\textless{}{-}} \FunctionTok{read\_csv}\NormalTok{(}\StringTok{\textquotesingle{}202508{-}divvy{-}tripdata.csv\textquotesingle{}}\NormalTok{)}
\NormalTok{aug }\OtherTok{\textless{}{-}} \FunctionTok{read\_csv}\NormalTok{(}\StringTok{\textquotesingle{}202507{-}divvy{-}tripdata.csv\textquotesingle{}}\NormalTok{)}
\NormalTok{jul }\OtherTok{\textless{}{-}} \FunctionTok{read\_csv}\NormalTok{(}\StringTok{\textquotesingle{}202506{-}divvy{-}tripdata.csv\textquotesingle{}}\NormalTok{)}
\NormalTok{jun }\OtherTok{\textless{}{-}} \FunctionTok{read\_csv}\NormalTok{(}\StringTok{\textquotesingle{}202505{-}divvy{-}tripdata.csv\textquotesingle{}}\NormalTok{)}
\NormalTok{may }\OtherTok{\textless{}{-}} \FunctionTok{read\_csv}\NormalTok{(}\StringTok{\textquotesingle{}202504{-}divvy{-}tripdata.csv\textquotesingle{}}\NormalTok{)}
\NormalTok{apr }\OtherTok{\textless{}{-}} \FunctionTok{read\_csv}\NormalTok{(}\StringTok{\textquotesingle{}202503{-}divvy{-}tripdata.csv\textquotesingle{}}\NormalTok{)}
\NormalTok{mar }\OtherTok{\textless{}{-}} \FunctionTok{read\_csv}\NormalTok{(}\StringTok{\textquotesingle{}202502{-}divvy{-}tripdata.csv\textquotesingle{}}\NormalTok{)}
\NormalTok{feb }\OtherTok{\textless{}{-}} \FunctionTok{read\_csv}\NormalTok{(}\StringTok{\textquotesingle{}202501{-}divvy{-}tripdata.csv\textquotesingle{}}\NormalTok{)}
\NormalTok{jan }\OtherTok{\textless{}{-}} \FunctionTok{read\_csv}\NormalTok{(}\StringTok{\textquotesingle{}202412{-}divvy{-}tripdata.csv\textquotesingle{}}\NormalTok{)}

\NormalTok{all\_trips }\OtherTok{\textless{}{-}} \FunctionTok{bind\_rows}\NormalTok{(jan, feb, mar, apr, may, jun, jul,aug, sep, oct, nov, dec)}
\end{Highlighting}
\end{Shaded}

\begin{itemize}
\tightlist
\item
  creating a ride\_length variable, removing zero-length rides,
  generating weekdays variables,
\end{itemize}

\begin{Shaded}
\begin{Highlighting}[]
\NormalTok{all\_trips }\OtherTok{\textless{}{-}}\NormalTok{ all\_trips }\SpecialCharTok{\%\textgreater{}\%}
     \FunctionTok{mutate}\NormalTok{(}\AttributeTok{ride\_length =}\NormalTok{ ended\_at }\SpecialCharTok{{-}}\NormalTok{ started\_at, }
        \AttributeTok{weekday =} \FunctionTok{wday}\NormalTok{(started\_at, }\AttributeTok{label =} \ConstantTok{TRUE}\NormalTok{)) }\SpecialCharTok{\%\textgreater{}\%}
     \FunctionTok{filter}\NormalTok{(ride\_length }\SpecialCharTok{\textgreater{}} \DecValTok{0}\NormalTok{)}
\end{Highlighting}
\end{Shaded}

\begin{itemize}
\tightlist
\item
  creating a variables that counts all rides by user type.
\end{itemize}

\begin{Shaded}
\begin{Highlighting}[]
\NormalTok{ride\_counts }\OtherTok{\textless{}{-}}\NormalTok{ all\_trips }\SpecialCharTok{\%\textgreater{}\%}
     \FunctionTok{count}\NormalTok{(member\_casual, weekday)}
\end{Highlighting}
\end{Shaded}

\subsection{Analyze}\label{analyze}

\subsubsection{Find trends and patterns}\label{find-trends-and-patterns}

Here I explore key statistics that relate directly to stakeholder
questions:

\begin{itemize}
\tightlist
\item
  ride length summaries,
\end{itemize}

\begin{Shaded}
\begin{Highlighting}[]
\NormalTok{ summary\_table }\OtherTok{\textless{}{-}}\NormalTok{ all\_trips }\SpecialCharTok{\%\textgreater{}\%}
     \FunctionTok{group\_by}\NormalTok{(member\_casual, weekday) }\SpecialCharTok{\%\textgreater{}\%}
     \FunctionTok{summarise}\NormalTok{(}\AttributeTok{mean\_length =} \FunctionTok{mean}\NormalTok{(ride\_length))}

\NormalTok{ max\_length }\OtherTok{\textless{}{-}} \FunctionTok{max}\NormalTok{(all\_trips}\SpecialCharTok{$}\NormalTok{ride\_length, }\AttributeTok{na.rm =} \ConstantTok{TRUE}\NormalTok{)}
 
\NormalTok{ min\_length }\OtherTok{\textless{}{-}} \FunctionTok{min}\NormalTok{(all\_trips}\SpecialCharTok{$}\NormalTok{ride\_length, }\AttributeTok{na.rm =} \ConstantTok{TRUE}\NormalTok{)}
\end{Highlighting}
\end{Shaded}

\begin{itemize}
\item
  max\_length gave 94494.01 seconds as result. Very questionable data.
  Possibly shows some issue.
\item
  min\_length pulled 0.046 seconds as result. Again can be indicator of
  some issue or system testing. Value not to be considered in the
  current task.
\item
  creating a function that determines the mode for the weekdays usage,
\end{itemize}

\begin{Shaded}
\begin{Highlighting}[]
\NormalTok{mode\_finder }\OtherTok{\textless{}{-}} \ControlFlowTok{function}\NormalTok{(data, var) \{}
\NormalTok{    data }\SpecialCharTok{\%\textgreater{}\%}
        \FunctionTok{count}\NormalTok{(\{\{ var \}\}) }\SpecialCharTok{\%\textgreater{}\%}
        \FunctionTok{filter}\NormalTok{(n }\SpecialCharTok{==} \FunctionTok{max}\NormalTok{(n)) }\SpecialCharTok{\%\textgreater{}\%}
        \FunctionTok{pull}\NormalTok{(\{\{ var \}\})}
\NormalTok{\}}


\FunctionTok{mode\_finder}\NormalTok{(all\_trips, weekday)}
\NormalTok{[}\DecValTok{1}\NormalTok{] Sat}
\NormalTok{Levels}\SpecialCharTok{:}\NormalTok{ Sun }\SpecialCharTok{\textless{}}\NormalTok{ Mon }\SpecialCharTok{\textless{}}\NormalTok{ Tue }\SpecialCharTok{\textless{}}\NormalTok{ Wed }\SpecialCharTok{\textless{}}\NormalTok{ Thu }\SpecialCharTok{\textless{}}\NormalTok{ Fri }\SpecialCharTok{\textless{}}\NormalTok{ Sat}
\end{Highlighting}
\end{Shaded}

Here we can see that saturday is the most busy day, all user-types
included.

\subsection{Share}\label{share}

\subsubsection{Make datasets easy to understand and show story through
visualizations}\label{make-datasets-easy-to-understand-and-show-story-through-visualizations}

\begin{itemize}
\item
  Created 3 different visualizations for stakeholders to see what the
  data tells.
\item
  All usages by weekdays.
\end{itemize}

\pandocbounded{\includegraphics[keepaspectratio]{images/sorted_by_weekday_all.png}}

\begin{itemize}
\tightlist
\item
  All usages splitted by member type.
\end{itemize}

\pandocbounded{\includegraphics[keepaspectratio]{images/sorted_by_weekday.png}}

\begin{itemize}
\tightlist
\item
  Side-by-side comparison as bar chart.
\end{itemize}

\pandocbounded{\includegraphics[keepaspectratio]{images/bar_chart_sorted.png}}

\subsection{Act}\label{act}

\subsubsection{Make some recommendations on what points should be seen,
changed or added to achieve business
goals}\label{make-some-recommendations-on-what-points-should-be-seen-changed-or-added-to-achieve-business-goals}

\begin{itemize}
\tightlist
\item
  Obviously members are more profitable quickly. Although they use bikes
  for longer time, which can be counter-productive as less bikes are
  available if users ride for longer. It can create a lack of
  opportunity. As for the longest ride it suggests either a bike got
  lost/stolen or is not working properly. Highly recommended to address
  this issue as it can bias the data and might tell about some issues
  with the product or system itself.
\end{itemize}

\end{document}
